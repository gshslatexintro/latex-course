\documentclass{beamer}

\input{preamble.tex}

\subtitle{제 1강: 기초}

\begin{document}

%%%%%%%%%%%%%%%%%%%%%%%%%%%%%%%%%%%%%%%%%%%%%%%%%%%%%%%%%%%%%%%%%%%%%%%%%%%%%%%
%%%%%%%%%%%%%%%%%%%%%%%%%%%%%%%%%%%%%%%%%%%%%%%%%%%%%%%%%%%%%%%%%%%%%%%%%%%%%%%
%%%%%%%%%%%%%%%%%%%%%%%%%%%%%%%%%%%%%%%%%%%%%%%%%%%%%%%%%%%%%%%%%%%%%%%%%%%%%%%
\begin{frame}
\titlepage
\end{frame}

%%%%%%%%%%%%%%%%%%%%%%%%%%%%%%%%%%%%%%%%%%%%%%%%%%%%%%%%%%%%%%%%%%%%%%%%%%%%%%%
%%%%%%%%%%%%%%%%%%%%%%%%%%%%%%%%%%%%%%%%%%%%%%%%%%%%%%%%%%%%%%%%%%%%%%%%%%%%%%%
%%%%%%%%%%%%%%%%%%%%%%%%%%%%%%%%%%%%%%%%%%%%%%%%%%%%%%%%%%%%%%%%%%%%%%%%%%%%%%%
\begin{frame}{\LaTeX{}을 사용하는 이유?}
\begin{itemize}
\item 깔끔한 문서를 만든다.
\begin{itemize}
\item 특히 수학에서
\end{itemize}
%
\item 과학자들에 의해, 과학자들을 위해 만들어졌다.
\begin{itemize}
\item 활동적인 큰 규모의 공동체
\end{itemize}
%
\item 여기저기에 확장시킬 수 있을 정도로 강력
\begin{itemize}
\item 논문, 발표자료, 스프레드시트, \ldots 을 위한 패키지들
\end{itemize}
\end{itemize}
\end{frame}

%%%%%%%%%%%%%%%%%%%%%%%%%%%%%%%%%%%%%%%%%%%%%%%%%%%%%%%%%%%%%%%%%%%%%%%%%%%%%%%
%%%%%%%%%%%%%%%%%%%%%%%%%%%%%%%%%%%%%%%%%%%%%%%%%%%%%%%%%%%%%%%%%%%%%%%%%%%%%%%
%%%%%%%%%%%%%%%%%%%%%%%%%%%%%%%%%%%%%%%%%%%%%%%%%%%%%%%%%%%%%%%%%%%%%%%%%%%%%%%
\begin{frame}[fragile]{작동 방식?}
\begin{itemize}
\item \texttt{순수 텍스트} 및 문서의 구조와 의미를 나타내는 \cmd{명령어}로
이루어진 문서를 작성하면 된다.
\item \texttt{latex} 프로그램은 작성된 문서의 텍스트와 명령어를 처리하여,
깔끔한 형태의 문서를 만든다.
\end{itemize}
\vskip 2ex
\begin{center}
\begin{lstlistings}
동해물과 백두산이 \emph{마르고} 닳도록
\end{lstlistings}
\vskip 2ex
\tikz\node[single arrow,fill=gray,font=\ttfamily\bfseries,%
  rotate=270,xshift=-1em]{latex};
\vskip 2ex
\fbox{동해물과 백두산이 \emph{마르고} 닳도록}
\end{center}
\end{frame}

%%%%%%%%%%%%%%%%%%%%%%%%%%%%%%%%%%%%%%%%%%%%%%%%%%%%%%%%%%%%%%%%%%%%%%%%%%%%%%%
%%%%%%%%%%%%%%%%%%%%%%%%%%%%%%%%%%%%%%%%%%%%%%%%%%%%%%%%%%%%%%%%%%%%%%%%%%%%%%%
%%%%%%%%%%%%%%%%%%%%%%%%%%%%%%%%%%%%%%%%%%%%%%%%%%%%%%%%%%%%%%%%%%%%%%%%%%%%%%%
\begin{frame}[fragile]{명령어와 결과물에 관한 또 다른 예시들\ldots}
\begin{exampletwoup}
\begin{itemize}
\item 차
\item 우유
\item 비스킷
\end{itemize}
\end{exampletwoup}
\vskip 2ex
\begin{exampletwoup}
\begin{figure}
\includegraphics{chick}
\end{figure}
\end{exampletwoup}
\vskip 2ex
\begin{exampletwoup}
\begin{equation}
\alpha + \beta + 1
\end{equation}
\end{exampletwoup}

\tiny{Image from \url{http://www.andy-roberts.net/writing/latex/importing_images}}
\end{frame}

%%%%%%%%%%%%%%%%%%%%%%%%%%%%%%%%%%%%%%%%%%%%%%%%%%%%%%%%%%%%%%%%%%%%%%%%%%%%%%%
%%%%%%%%%%%%%%%%%%%%%%%%%%%%%%%%%%%%%%%%%%%%%%%%%%%%%%%%%%%%%%%%%%%%%%%%%%%%%%%
%%%%%%%%%%%%%%%%%%%%%%%%%%%%%%%%%%%%%%%%%%%%%%%%%%%%%%%%%%%%%%%%%%%%%%%%%%%%%%%
\begin{frame}[fragile]{태도 변화}

\begin{itemize}
\item `어떻게 보이는지'가 아닌 `어떤 것인지'를 설명하기 위해 명령어 사용
\item 내용에 집중하라.
\item \LaTeX{}이 `어떻게 보이는지'에 관한 일을 하게끔 하라.
\end{itemize}
\end{frame}

%%%%%%%%%%%%%%%%%%%%%%%%%%%%%%%%%%%%%%%%%%%%%%%%%%%%%%%%%%%%%%%%%%%%%%%%%%%%%%%
%%%%%%%%%%%%%%%%%%%%%%%%%%%%%%%%%%%%%%%%%%%%%%%%%%%%%%%%%%%%%%%%%%%%%%%%%%%%%%%
%%%%%%%%%%%%%%%%%%%%%%%%%%%%%%%%%%%%%%%%%%%%%%%%%%%%%%%%%%%%%%%%%%%%%%%%%%%%%%%
\section{기초}

%%%%%%%%%%%%%%%%%%%%%%%%%%%%%%%%%%%%%%%%%%%%%%%%%%%%%%%%%%%%%%%%%%%%%%%%%%%%%%%
%%%%%%%%%%%%%%%%%%%%%%%%%%%%%%%%%%%%%%%%%%%%%%%%%%%%%%%%%%%%%%%%%%%%%%%%%%%%%%%
%%%%%%%%%%%%%%%%%%%%%%%%%%%%%%%%%%%%%%%%%%%%%%%%%%%%%%%%%%%%%%%%%%%%%%%%%%%%%%%
\subsection{시작하기}
\begin{frame}[fragile]{\insertsubsection}
\begin{itemize}
\item 최소의 \LaTeX{} 문서:
\inputminted[frame=single]{latex}{basics.tex}
\item 모든 명령어는 \emph{backslash} \keystrokebftt{\bs} 로 시작한다.
\item 모든 문서는 \cmdbs{documentclass} 로 시작한다.
\item 중괄호 \keystrokebftt{\{} \keystrokebftt{\}} 안의 \emph{argument}는
우리가 어떤 종류의 문서를 만드는 것인지 \LaTeX{} 에게 알려준다: 이 경우 \bftt{article}.
\item 퍼센트 기호 \keystrokebftt{\%} 는 \emph{주석}이다 --- \LaTeX{}은
주석 뒷부분의 텍스트를 모두 무시한다.
\end{itemize}
\end{frame}

%%%%%%%%%%%%%%%%%%%%%%%%%%%%%%%%%%%%%%%%%%%%%%%%%%%%%%%%%%%%%%%%%%%%%%%%%%%%%%%
%%%%%%%%%%%%%%%%%%%%%%%%%%%%%%%%%%%%%%%%%%%%%%%%%%%%%%%%%%%%%%%%%%%%%%%%%%%%%%%
%%%%%%%%%%%%%%%%%%%%%%%%%%%%%%%%%%%%%%%%%%%%%%%%%%%%%%%%%%%%%%%%%%%%%%%%%%%%%%%
\begin{frame}[fragile]{\insertsubsection{} with \wllogo}
\begin{itemize}
\item Overleaf 는 \LaTeX 으로 문서를 작성하는 웹사이트다.
\item 이는 \LaTeX{} 을 자동으로 조판하여 결과를 보여준다.
\vskip 2em
\begin{center}
\fbox{\href{\wlnewdoc{basics.tex}}{%
\wllogo{}에서 예시 문서를 열기 위해 이곳을 누르시오}}
\\[1ex]\scriptsize{}
최선의 품질을 위해, \href{http://www.google.com/chrome}{Google Chrome}
또는 최신의 \href{http://www.mozilla.org/en-US/firefox/new/}{FireFox} 브라우저를 사용해 주십시오.
\end{center}
\vskip 2ex
\item 다음의 슬라이드들을 보며, Overleaf에서 예시 문서를 열어
그 예시들을 직접 쳐보고 조판해보시오.
\item \textbf{직접 해봐야 진전이 이루어질 수 있습니다!}
\end{itemize}
\end{frame}

%%%%%%%%%%%%%%%%%%%%%%%%%%%%%%%%%%%%%%%%%%%%%%%%%%%%%%%%%%%%%%%%%%%%%%%%%%%%%%%
%%%%%%%%%%%%%%%%%%%%%%%%%%%%%%%%%%%%%%%%%%%%%%%%%%%%%%%%%%%%%%%%%%%%%%%%%%%%%%%
%%%%%%%%%%%%%%%%%%%%%%%%%%%%%%%%%%%%%%%%%%%%%%%%%%%%%%%%%%%%%%%%%%%%%%%%%%%%%%%
\subsection{텍스트 조판하기}
\begin{frame}[fragile]{\insertsubsection{}}
\small
\begin{itemize}
\item \cmdbegin{document} 와 \cmdend{document} 사이에 텍스트를 입력하십시오.
\item 대부분의 경우 텍스트를 평소처럼 입력할 수 있습니다.
\begin{exampletwouptiny}
단어들은 한개 또는 그 이상의 공백으로
구분된다.

문단들은 한개 또는 그 이상의 개행으로
구분된다.
\end{exampletwouptiny}
\item 코드 파일에서의 공백은 결과물에서 하나로 합쳐집니다.
\begin{exampletwouptiny}
무궁화   삼천리   화려강산
대한사람 대한으로 길이 보전하세
\end{exampletwouptiny}
\end{itemize}
\end{frame}

%%%%%%%%%%%%%%%%%%%%%%%%%%%%%%%%%%%%%%%%%%%%%%%%%%%%%%%%%%%%%%%%%%%%%%%%%%%%%%%
%%%%%%%%%%%%%%%%%%%%%%%%%%%%%%%%%%%%%%%%%%%%%%%%%%%%%%%%%%%%%%%%%%%%%%%%%%%%%%%
%%%%%%%%%%%%%%%%%%%%%%%%%%%%%%%%%%%%%%%%%%%%%%%%%%%%%%%%%%%%%%%%%%%%%%%%%%%%%%%
\begin{frame}[fragile]{\insertsubsection{}: 경고}
\small
\begin{itemize}
\item 따옴표는 쓰기가 약간 어렵습니다: 앞쪽은 backtick \keystroke{\`{}}을, 뒤쪽은 apostrophe \keystroke{\'{}} 을 써야 한다.
\begin{exampletwouptiny}
작은따옴표: `텍스트'.

큰따옴표: ``텍스트''.
\end{exampletwouptiny}

\item 몇 개의 특수문자는 \LaTeX 의 예약어입니다:\\[1ex]
\begin{tabular}{cl}
\keystrokebftt{\%} & 퍼센트 기호               \\
\keystrokebftt{\#} & 샾(우물 정 자)            \\
\keystrokebftt{\&} & ampersand                 \\
\keystrokebftt{\$} & 달러 기호                 \\
\end{tabular}
\item 이 기호들은 그대로 입력한다면 에러가 발생할 것이다. 이 기호들을 결과로써
나타나게 하고 싶다면, 앞쪽에 backslash를 붙여서 \emph{escape} 해야 한다.
\begin{exampletwoup}
\$\%\&\#!
\end{exampletwoup}
\end{itemize}
\end{frame}

%%%%%%%%%%%%%%%%%%%%%%%%%%%%%%%%%%%%%%%%%%%%%%%%%%%%%%%%%%%%%%%%%%%%%%%%%%%%%%%
%%%%%%%%%%%%%%%%%%%%%%%%%%%%%%%%%%%%%%%%%%%%%%%%%%%%%%%%%%%%%%%%%%%%%%%%%%%%%%%
%%%%%%%%%%%%%%%%%%%%%%%%%%%%%%%%%%%%%%%%%%%%%%%%%%%%%%%%%%%%%%%%%%%%%%%%%%%%%%%
\begin{frame}[fragile]{오류 손보기}
\begin{itemize}
\item \LaTeX{} 은 문서를 조판하는 동안 혼란스러워(confused) 할 수 있다.
만약 그럴 경우, 에러와 함께 조판이 중단된다. 이러한 에러는 반드시 결과물이 나오기 전에
수정되어야만 한다.
\item 예를 들어, \cmdbs{emph} 를 \cmdbs{meph} 로 잘못 쳤다면, \LaTeX{} 은
``undefined control sequence'' 오류와 함께 멈출 것이다. ``meph'' 는 지정되지 않은
명령어이기 때문이다.
\end{itemize}
\begin{block}{오류에 관한 조언}
\begin{enumerate}
\item 당황할 것 없다. 오류 발생은 빈번할 것이다.
\item 오류가 발생한다면, 그와 동시에 수정을 시작하라 --- 입력한 코드가 오류를
야기한다면, 그 시점에서 디버깅을 시작하면 된다.
\item 여러 개의 오류가 발생한다면, 우선 첫 번째 것부터 고쳐라 --- 오류 원인은
한 곳에서 발생한 것일 수 있다.
% TODO 어색한 번역일지도 모르겠습니다. 원문은 아래와 같으니, 더 나은 번역이 있다면 고쳐 주십시오.
% If there are multiple errors, start with the first one --- the cause may even be above it.
\end{enumerate}
\end{block}
\end{frame}

%%%%%%%%%%%%%%%%%%%%%%%%%%%%%%%%%%%%%%%%%%%%%%%%%%%%%%%%%%%%%%%%%%%%%%%%%%%%%%%
%%%%%%%%%%%%%%%%%%%%%%%%%%%%%%%%%%%%%%%%%%%%%%%%%%%%%%%%%%%%%%%%%%%%%%%%%%%%%%%
%%%%%%%%%%%%%%%%%%%%%%%%%%%%%%%%%%%%%%%%%%%%%%%%%%%%%%%%%%%%%%%%%%%%%%%%%%%%%%%
\begin{frame}[fragile]{연습문제 1 조판하기}

\begin{block}{다음을 \LaTeX 으로 조판하시오:
%TODO 한국어 번역본에 걸맞게 한국어로 된 위키백과 문서 중 따옴표, $, % 기호를 포함하는 부분을 찾아서 넣었으면 좋겠습니다.
\footnote{\url{http://en.wikipedia.org/wiki/Economy_of_the_United_States}}}
In March 2006, Congress raised that ceiling an additional \$0.79 trillion to
\$8.97 trillion, which is approximately 68\% of GDP. As of October 4, 2008, the
``Emergency Economic Stabilization Act of 2008'' raised the current debt ceiling
to \$11.3 trillion.
\end{block}
\vskip 2ex
\begin{center}
\fbox{\href{\wlnewdoc{basics-exercise-1.tex}}{%
\wllogo{}에서 예시 문서를 열기 위해 이곳을 누르시오}}
\end{center}

\begin{itemize}
\item 힌트: 특수문자의 사용에 유의하십시오!
\item 성공한 것 같다면,
\fbox{\href{\wlnewdoc{basics-exercise-1-solution.tex}}{%
나의 답을 보기 위해 여기를 클릭하라}}.
\end{itemize}
\end{frame}

%%%%%%%%%%%%%%%%%%%%%%%%%%%%%%%%%%%%%%%%%%%%%%%%%%%%%%%%%%%%%%%%%%%%%%%%%%%%%%%
%%%%%%%%%%%%%%%%%%%%%%%%%%%%%%%%%%%%%%%%%%%%%%%%%%%%%%%%%%%%%%%%%%%%%%%%%%%%%%%
%%%%%%%%%%%%%%%%%%%%%%%%%%%%%%%%%%%%%%%%%%%%%%%%%%%%%%%%%%%%%%%%%%%%%%%%%%%%%%%
\subsection{수식 조판하기}
\begin{frame}[fragile]{\insertsubsection{}: 달러 기호}
\begin{itemize}
\item 달러 기호 \keystrokebftt{\$} 는 뭐가 그리 특별한가? 이것은 수식을 표시하기 위해 쓰인다.\\[1ex]
\begin{exampletwouptiny}
% 좋지 않은 예시:
서로 다른 양의 정수 a, b에 대해
c = a - b + 1 를 정의하자.

% 훨씬 좋은 예시:
서로 다른 양의 정수 $a$, $b$에 대해
$c = a - b + 1$ 를 정의하자.
\end{exampletwouptiny}
\item 달러 기호를 반드시 쌍으로 써야 한다 --- 하나는 수식을 열고, 다른 하나는 닫는 데에.
\item \LaTeX{} 은 공백을 자동적으로 조절한다; 코드의 공백을 무시한다.
\begin{exampletwouptiny}
$y=mx+b$ 를 \ldots 와 같이 정의하자.

$y = m x + b$ 를 \ldots 와 같이 정의하자.
\end{exampletwouptiny}
\end{itemize}
\end{frame}

%%%%%%%%%%%%%%%%%%%%%%%%%%%%%%%%%%%%%%%%%%%%%%%%%%%%%%%%%%%%%%%%%%%%%%%%%%%%%%%
%%%%%%%%%%%%%%%%%%%%%%%%%%%%%%%%%%%%%%%%%%%%%%%%%%%%%%%%%%%%%%%%%%%%%%%%%%%%%%%
%%%%%%%%%%%%%%%%%%%%%%%%%%%%%%%%%%%%%%%%%%%%%%%%%%%%%%%%%%%%%%%%%%%%%%%%%%%%%%%
\begin{frame}[fragile]{\insertsubsection{}: 표기 방법들}
\begin{itemize}
\item 위첨자는 caret \keystrokebftt{\^} 을, 아래첨자는 underscore \keystrokebftt{\_} 를 사용하라.
\begin{exampletwouptiny}
$y = c_2 x^2 + c_1 x + c_0$
\end{exampletwouptiny}
\vskip 2ex

\item 위첨자 또는 아래첨자를 묶기 위해 중괄호 \keystrokebftt{\{} \keystrokebftt{\}} 를 사용하라.
\begin{exampletwouptiny}
$F_n = F_n-1 + F_n-2$     % 저런!

$F_n = F_{n-1} + F_{n-2}$ % 옳지!
\end{exampletwouptiny}
\vskip 2ex

\item 그리스 문자나 각종 표기법을 표현하는 명령어도 있다.
\begin{exampletwouptiny}
$\mu = A e^{Q/RT}$

$\Omega = \sum_{k=1}^{n} \omega_k$
\end{exampletwouptiny}
\end{itemize}
\end{frame}

%%%%%%%%%%%%%%%%%%%%%%%%%%%%%%%%%%%%%%%%%%%%%%%%%%%%%%%%%%%%%%%%%%%%%%%%%%%%%%%
%%%%%%%%%%%%%%%%%%%%%%%%%%%%%%%%%%%%%%%%%%%%%%%%%%%%%%%%%%%%%%%%%%%%%%%%%%%%%%%
%%%%%%%%%%%%%%%%%%%%%%%%%%%%%%%%%%%%%%%%%%%%%%%%%%%%%%%%%%%%%%%%%%%%%%%%%%%%%%%
\begin{frame}[fragile]{\insertsubsection{}: Displayed Equations}
\begin{itemize}
\item 만약 수식이 너무 크다면, \cmdbegin{equation} 와 \cmdend{equation}를 통해
단독의 줄에 이를 \emph{display} 하라.\\[2ex]
\begin{exampletwouptiny}
$a$, $b$, 그리고 $c$ 가 \ldots 일 때,
이차방정식의 근은
\begin{equation}
x = \frac{-b \pm \sqrt{b^2 - 4ac}}
         {2a}
\end{equation}
와 같이 주어진다.
\end{exampletwouptiny}
\vskip 1em
{\scriptsize 주의: 웬만하면 \LaTeX{} 은 수식 코드에서의 공백은 무시하지만,
빈 줄을 무시해내지는 못한다 --- 수식 코드에 빈 줄을 넣지 말아야 한다.}
\end{itemize}
\end{frame}

%%%%%%%%%%%%%%%%%%%%%%%%%%%%%%%%%%%%%%%%%%%%%%%%%%%%%%%%%%%%%%%%%%%%%%%%%%%%%%%
%%%%%%%%%%%%%%%%%%%%%%%%%%%%%%%%%%%%%%%%%%%%%%%%%%%%%%%%%%%%%%%%%%%%%%%%%%%%%%%
%%%%%%%%%%%%%%%%%%%%%%%%%%%%%%%%%%%%%%%%%%%%%%%%%%%%%%%%%%%%%%%%%%%%%%%%%%%%%%%
\begin{frame}[fragile]{여담 : 환경(environments)}
\begin{itemize}
\item \bftt{수식}은 \emph{환경}이다.
\item 동일한 명령어도 환경에 따라 결과가 달라진다.
\begin{exampletwouptiny}
우리는 텍스트에서는
$ \Omega = \sum_{k=1}^{n} \omega_k $
와 같이 쓸 수 있고, 아니면
\begin{equation}
  \Omega = \sum_{k=1}^{n} \omega_k
\end{equation}
와 같이 display할 수 있다.
\end{exampletwouptiny}
\vskip 2ex
\item 동일한 코드임에도 불구하고 $\Sigma$가 \bftt{equation} 환경에서 더 커진 것과, 
위첨자 또는 아래첨자가 달라진 것에 대해 주목하라.
\vskip 1em
{\scriptsize 사실, 우리는 \bftt{\$...\$} 를
\cmdbegin{math}\bftt{...}\cmdend{math}로 쓸 수도 있다.}
\end{itemize}
\end{frame}

%%%%%%%%%%%%%%%%%%%%%%%%%%%%%%%%%%%%%%%%%%%%%%%%%%%%%%%%%%%%%%%%%%%%%%%%%%%%%%%
%%%%%%%%%%%%%%%%%%%%%%%%%%%%%%%%%%%%%%%%%%%%%%%%%%%%%%%%%%%%%%%%%%%%%%%%%%%%%%%
%%%%%%%%%%%%%%%%%%%%%%%%%%%%%%%%%%%%%%%%%%%%%%%%%%%%%%%%%%%%%%%%%%%%%%%%%%%%%%%
\begin{frame}[fragile]{여담 : 환경(environments)}
\begin{itemize}
\item \cmdbs{begin} 와 \cmdbs{end} 는 다양한 종류의 환경을 만들어내는 데에 사용된다.
\vskip 2ex

\item \bftt{itemize} 와 \bftt{enumerate} 환경은 목록을 만들어낼 때 사용된다.
\begin{exampletwouptiny}
\begin{itemize} % 점을 찍어서 나열
\item 비스킷
\item 차
\end{itemize}

\begin{enumerate} % 번호를 매겨서 나열
\item 비스킷
\item 차
\end{enumerate}
\end{exampletwouptiny}
\end{itemize}
\end{frame}

%%%%%%%%%%%%%%%%%%%%%%%%%%%%%%%%%%%%%%%%%%%%%%%%%%%%%%%%%%%%%%%%%%%%%%%%%%%%%%%
%%%%%%%%%%%%%%%%%%%%%%%%%%%%%%%%%%%%%%%%%%%%%%%%%%%%%%%%%%%%%%%%%%%%%%%%%%%%%%%
%%%%%%%%%%%%%%%%%%%%%%%%%%%%%%%%%%%%%%%%%%%%%%%%%%%%%%%%%%%%%%%%%%%%%%%%%%%%%%%
\begin{frame}[fragile]{여담: 패키지(packages)}

\begin{itemize}
\item 그동안 우리가 써왔던 명령어 및 환경은 \LaTeX 으로 빌드된다.

\item \emph{패키지}는 이외의 명령어나 환경을 정의하는 라이브러리다.
무료로 사용 가능한 수천 개의 패키지들이 공개되어 있다.

\item 이러한 패키지를 불러오기 위해서는, 우리는 \emph{preamble} 안에
\cmdbs{usepackage} 명령어를 사용해야 한다.

\item 예시: 미국수학회의 \bftt{amsmath}.
\begin{lstlistings}
\documentclass{article}
\usepackage{amsmath} % preamble
\begin{document}
% 이제 여기에 amsmath에 정의된 명령어를 사용할 수 있다.
\end{document}
\end{lstlistings}
\end{itemize}
\end{frame}

%%%%%%%%%%%%%%%%%%%%%%%%%%%%%%%%%%%%%%%%%%%%%%%%%%%%%%%%%%%%%%%%%%%%%%%%%%%%%%%
%%%%%%%%%%%%%%%%%%%%%%%%%%%%%%%%%%%%%%%%%%%%%%%%%%%%%%%%%%%%%%%%%%%%%%%%%%%%%%%
%%%%%%%%%%%%%%%%%%%%%%%%%%%%%%%%%%%%%%%%%%%%%%%%%%%%%%%%%%%%%%%%%%%%%%%%%%%%%%%
\begin{frame}[fragile]{\insertsubsection{}: \bftt{amsmath}에 관한 예시}
\begin{itemize}
\item 번호가 없는 수식은 \bftt{equation*} (``equation-star'') 를 사용하라.
\begin{exampletwouptiny}
\begin{equation*}
  \Omega = \sum_{k=1}^{n} \omega_k
\end{equation*}
\end{exampletwouptiny}
\item \LaTeX{} 은 인접한 문자들을 마치 변수들의 곱으로 취급하고, 이는 항상 옳지는 않다.
\bftt{amsmath} 는 많은 종류의 수학적 연산자 및 표기법을 정의한다.
mathematical operators.
\begin{exampletwouptiny}
\begin{equation*} % 안 좋은 예시!
 min_{x,y} (1-x)^2 + 100(y-x^2)^2
\end{equation*}
\begin{equation*} % 좋은 예시!
\min_{x,y}{(1-x)^2 + 100(y-x^2)^2}
\end{equation*}
\end{exampletwouptiny}
\item 그 외의 경우 \cmdbs{operatorname} 를 사용할 수 있다.
\begin{exampletwouptiny}
\begin{equation*}
\beta_i =
\frac{\operatorname{Cov}(R_i, R_m)}
     {\operatorname{Var}(R_m)}
\end{equation*}
\end{exampletwouptiny}
\end{itemize}
\end{frame}

%%%%%%%%%%%%%%%%%%%%%%%%%%%%%%%%%%%%%%%%%%%%%%%%%%%%%%%%%%%%%%%%%%%%%%%%%%%%%%%
%%%%%%%%%%%%%%%%%%%%%%%%%%%%%%%%%%%%%%%%%%%%%%%%%%%%%%%%%%%%%%%%%%%%%%%%%%%%%%%
%%%%%%%%%%%%%%%%%%%%%%%%%%%%%%%%%%%%%%%%%%%%%%%%%%%%%%%%%%%%%%%%%%%%%%%%%%%%%%%
\begin{frame}[fragile]{\insertsubsection{}: \bftt{amsmath}에 관한 예시}
\begin{itemize}{\small
\item 이어져 있는 등식들을 등호를 기준으로 정렬하려면
\begin{align*}
(x+1)^3 &= (x+1)(x+1)(x+1) \\
        &= (x+1)(x^2 + 2x + 1) \\
        &= x^3 + 3x^2 + 3x + 1
\end{align*}
와 같이 \bftt{align*} 환경을 사용하라.

% for whatever reason, this doesn't play well with the twoup environment
\begin{lstlistings}[fontsize=\small,frame=single]{latex}
\begin{align*}
(x+1)^3 &= (x+1)(x+1)(x+1) \\
        &= (x+1)(x^2 + 2x + 1) \\
        &= x^3 + 3x^2 + 3x + 1
\end{align*}
\end{lstlistings}
\item Ampersand \keystrokebftt{\&} 는 왼쪽 열($=$ 이전)과 오른쪽 열($=$ 이후)를 구분짓는다.
\item 두 개의 backslash \keystrokebftt{\bs}\keystrokebftt{\bs} 는 개행을 한다.
}\end{itemize}
\end{frame}


%%%%%%%%%%%%%%%%%%%%%%%%%%%%%%%%%%%%%%%%%%%%%%%%%%%%%%%%%%%%%%%%%%%%%%%%%%%%%%%
%%%%%%%%%%%%%%%%%%%%%%%%%%%%%%%%%%%%%%%%%%%%%%%%%%%%%%%%%%%%%%%%%%%%%%%%%%%%%%%
%%%%%%%%%%%%%%%%%%%%%%%%%%%%%%%%%%%%%%%%%%%%%%%%%%%%%%%%%%%%%%%%%%%%%%%%%%%%%%%
\begin{frame}[fragile]{연습문제 2 조판하기}

\begin{block}{다음을 \LaTeX 으로 조판하시오:}
Let $X_1, X_2, \ldots, X_n$ be a sequence of independent and identically
distributed random variables with $\operatorname{E}[X_i] = \mu$ and
$\operatorname{Var}[X_i] = \sigma^2 < \infty$, and let
\begin{equation*}
S_n = \frac{1}{n}\sum_{i}^{n} X_i
\end{equation*}
denote their mean. Then as $n$ approaches infinity, the random variables
$\sqrt{n}(S_n - \mu)$ converge in distribution to a normal $N(0, \sigma^2)$.
\end{block}
\vskip 2ex
\begin{center}
\fbox{\href{\wlnewdoc{basics-exercise-2.tex}}{%
\wllogo{}에서 예시 문서를 열기 위해 이곳을 누르시오}}
\end{center}
\begin{itemize}
\item 힌트: $\infty$ 를 나타내는 명령어는 \cmdbs{infty} 이다.
\item 성공한 것 같다면,
\fbox{\href{\wlnewdoc{basics-exercise-2-solution.tex}}{%
나의 답을 보기 위해 여기를 클릭하라}}.
\end{itemize}
\end{frame}

%%%%%%%%%%%%%%%%%%%%%%%%%%%%%%%%%%%%%%%%%%%%%%%%%%%%%%%%%%%%%%%%%%%%%%%%%%%%%%%
%%%%%%%%%%%%%%%%%%%%%%%%%%%%%%%%%%%%%%%%%%%%%%%%%%%%%%%%%%%%%%%%%%%%%%%%%%%%%%%
%%%%%%%%%%%%%%%%%%%%%%%%%%%%%%%%%%%%%%%%%%%%%%%%%%%%%%%%%%%%%%%%%%%%%%%%%%%%%%%
\begin{frame}{제 1강의 끝}
\begin{itemize}
\item 축하합니다! 당신은 이제
\begin{itemize}
\item \LaTeX 으로 텍스트를 조판하고,
\item 여러가지의 명령어를 사용하고,
\item 오류가 발생할 때 그를 고치고,
\item 깔끔한 수식을 조판하고,
\item 여러가지의 환경을 사용하며,
\item 패키지를 불러오는 것에 대해 알게 되었습니다.
\end{itemize}
\item 금세 이렇게나 많이 했다니!
\item 제 2강에서는, \LaTeX{}을 절(sections), cross references, 그림, 표, 그리고 참고문헌 등이 포함된
구조적인 문서를 작성할 때 사용하는 방법을 배울 것입니다. 그 때 만나요!
\end{itemize}
\end{frame}

\end{document}
